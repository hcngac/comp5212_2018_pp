\documentclass[a4paper]{article}

\usepackage[english]{babel}
\usepackage[utf8]{inputenc}
\usepackage{amsmath}
\usepackage{graphicx}

\title{COMP5212 Machine Learning 2018 Fall programming project proposal \\ CHESS AI}

\author{Hok Chun Ng, 20272532, hcngac@connect.ust.hk \\
        Shengyuan Zhang, 20565161, szhangcg@connect.ust.hk \\
        Ge Chen, 20360858, gchenaj@connect.ust.hk}

\date{\today}

\begin{document}
\maketitle

\begin{abstract}
In this project proposal, we are going to introduce the
problem related to chess AI and the method Q-learning applied
to solve the problem.
\end{abstract}

\section{Topic}

TODO


\section{Description of the application and justification for its practical significance}

TODO

\section{Formulation of the machine learning problems involved in the application}

TODO

\section{Data set (and preprocessing, if needed)}

TODO

\section{Machine learning methods}

TODO

\section{Design of experiments and performance evaluation}

TODO

% \begin{figure}
% \centering
% \includegraphics[width=0.3\textwidth]{frog.jpg}
% \caption{\label{fig:frog}This frog was uploaded to writeLaTeX via the project menu.}
% \end{figure}


% \begin{thebibliography}{9}
% \bibitem{nano3}
%   K. Grove-Rasmussen og Jesper Nygård,
%   \emph{Kvantefænomener i Nanosystemer}.
%   Niels Bohr Institute \& Nano-Science Center, Københavns Universitet
%
% \end{thebibliography}
\end{document}
